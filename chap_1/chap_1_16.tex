\documentclass[../main.tex]{subfiles}

\begin{document}

    \subsection{Gradient Operator}
    Looking at coordinate transformations on a 3-D coordinate system. \\
    Under a coordinate transformation that transforms $x_i$ into $x_i'$, a scalar $\phi$ remains invariant. 
    \begin{eqnindent}
        \begin{flalign}
            \phi\paren{x_1,~x_2,~x_3} = \phi'\paren{x_1',~x_2',~x_3'} &&
        \end{flalign}
    \end{eqnindent}
    The partial derivatives of $\phi'$ can be expanded using chain rule as
    \begin{eqnindent}
        \begin{flalign}
            \Rightarrow \frac{\partial\phi'}{\partial x_1'} = \sum_j\frac{\partial\phi}{\partial x_j}\frac{\partial x_j}{\partial x_1'},\quad\frac{\partial\phi'}{\partial x_2'} = \sum_j\frac{\partial\phi}{\partial x_j}\frac{\partial x_j}{\partial x_2'}\quad\&\quad\frac{\partial\phi'}{\partial x_3'} = \sum_j\frac{\partial\phi}{\partial x_j}\frac{\partial x_j}{\partial x_3'} &&
        \end{flalign}
    \end{eqnindent}
    which in summation notation surmises to
    \begin{eqnindent}
        \begin{flalign}
            \frac{\partial\phi'}{\partial x_i'} = \sum_j\frac{\partial\phi}{\partial x_j}\frac{\partial x_j}{\partial x_i'} &&
        \end{flalign}
    \end{eqnindent}
    To transform $x'$ to $x$, 
    \begin{eqnindent}
        \begin{flalign}
            x_j = \sum_k\lambda_{kj}x_k' &&
        \end{flalign}
    \end{eqnindent}
    with partial derivatives - 
    \begin{eqnindent}
        \begin{flalign}
            \frac{\partial x_j}{\partial x_i'} = \frac{\partial}{\partial x_i'}\paren{\sum_k\lambda_{kj}x_k'} &= \sum_k\lambda_{kj}\paren{\frac{\partial x_k'}{\partial x_i'}},\quad\text{where }\frac{\partial x_k'}{\partial x_i'} \equiv \delta_{ik} &&\nonumber\\
            &= \sum_k\lambda_{kj}\delta_{ik} &&\nonumber\\
            &= \lambda_{ij}
        \end{flalign}
    \end{eqnindent}
    which gives the partial derivatives of $\phi'$ as
    \begin{eqnindent}
        \begin{flalign}
            \frac{\partial\phi'}{\partial x_i'} = \sum_j\lambda_{ij}\frac{\partial\phi}{\partial x_j} &&
        \end{flalign}
    \end{eqnindent}
    showing that $\frac{\partial\phi}{\partial x_j}$ transforms as components of a vector. \\
    Vector gradient operator ($\bm{\nabla}$) expressed as
    \begin{eqnindent}
        \begin{flalign}
            \mathbf{grad} = \bm{\nabla} = \sum_i\bm{e}_i\frac{\partial}{\partial x_i} &&
        \end{flalign}
    \end{eqnindent}
    \begin{itemize}
        \renewcommand\labelitemi{--}
        \item can operate directly on a scalar function
        \begin{eqnindent}
            \begin{flalign}
                \mathbf{grad}~\phi = \bm{\nabla}\phi = \sum_i\bm{e}_i\frac{\partial\phi}{\partial x_i} &&
            \end{flalign}
        \end{eqnindent}
        Consider 3-D topographical maps as in \texttt{(Figure 1.16-1)}\\
        Let the scalar $\phi$ denote the height at any point $\phi = \phi\paren{x_1,~x_2,~x_3}$. \\
        An infinitesimal change in $\phi$ ($d\phi$) is
        \begin{eqnindent}
            \begin{flalign}
                d\phi = \sum_i\frac{\partial\phi}{\partial x_i}dx_i &= \sum_i\paren{\bm{\nabla}\phi}_idx_i &&\nonumber\\
                &= \paren{\bm{\nabla}\phi} \cdot d\bm{s},\quad\text{where }d\bm{s} = \paren{dx_1,~dx_2,~dx_3} &&
            \end{flalign}
        \end{eqnindent}
        \begin{enumerate}
            \item $\bm{\nabla}\phi$ is normal to lines or surfaces for which $\phi$ is constant
            \item $\bm{\nabla}\phi$ is in the direction of the maximum change in $\phi$
            \item the rate of change of $\phi$ in the direction of a unit vector $\bm{n}$ (directional derivative of $\phi$) is
            \begin{eqnindent}
                \begin{flalign}
                    \bm{n} \cdot \bm{\nabla}\phi \equiv \frac{\partial\phi}{\partial n} &&
                \end{flalign}
            \end{eqnindent}
        \end{enumerate}
        \item can be used in a scalar product with a vector function (divergence (div) of $\bm{A}$)
        \begin{eqnindent}
            \begin{flalign}
                \mathbf{div}~\bm{A} = \bm{\nabla} \cdot \bm{A} = \sum_i\bm{e}_i\frac{\partial A_i}{\partial x_i} &&
            \end{flalign}
        \end{eqnindent}
        \item can be used in a vector product with a vector function (curl of $\bm{A}$)
        \begin{eqnindent}
            \begin{flalign}
                \mathbf{curl}~\bm{A} = \bm{\nabla} \times \bm{A} = \sum_{i, j, k}\bm{e}_{ijk}\frac{\partial A_k}{\partial x_j}\bm{e}_i &&
            \end{flalign}
        \end{eqnindent}
    \end{itemize}
    Laplacian - successive operation of the gradient operator
    \begin{eqnindent}
        \begin{flalign}
            \bm{\nabla} \cdot \bm{\nabla} &= \sum_i\frac{\partial}{\partial x_i}\frac{\partial}{\partial x_i} = \sum_i\frac{\partial^2}{\partial x_i^2} &&\nonumber\\
            \bm{\nabla}^2 &= \sum_i\frac{\partial^2}{\partial x_i^2} &&
        \end{flalign}
    \end{eqnindent}
    \begin{indented}
        e.g. $\bm{\nabla}^2\psi = \sum_i\frac{\partial^2\phi}{\partial x_i^2}$
    \end{indented}

\end{document}
