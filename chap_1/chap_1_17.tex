\documentclass[../main.tex]{subfiles}

\begin{document}

    \subsection{Integration of Vectors}
    Volume integration of a vector function $\bm{A} = \bm{A}\paren{x_i}$ through a volume $V$
    \begin{eqnindent}
        \begin{flalign}
            \int_V\bm{A}dv = \paren{\int_VA_1dv\int_VA_2dv\int_VA_3dv} &&
        \end{flalign}
    \end{eqnindent}
    Consider a surface $S$ in 3-D coordinate system as in \texttt{(Figure 1.17-1)}\newline
    An integral over $S$ of the projection of a vector function $\bm{A} = \bm{A}\paren{x_i}$ onto the normal of the surface
    \begin{eqnindent}
        \begin{flalign}
            \int_S\bm{A} \cdot d\bm{a} &&
        \end{flalign}
    \end{eqnindent}
    where $d\bm{a}$ is an element of area of $S$
    \begin{dasheditemize}
        \item with magnitude $da$
        \item with direction normal to the surface\newline
        e.g. for the unit normal vector $\bm{n}$
        \begin{eqnindent}
            \begin{flalign}
                d\bm{a} = \bm{n}da &&
            \end{flalign}
        \end{eqnindent}
        for a closed surface, the \textit{outward} normal is taken as the positive direction following conventions
    \end{dasheditemize}
    Similarly, the components of $d\bm{a}$ are projections of the element of area on the three mutually perpentually planes defined by the rectangular axes
    \begin{indented}
        e.g. $da_1 = dx_2dx_3,\quad da_2 = dx_1dx_3\quad\&\quad da_3 = dx_1dx_2$
    \end{indented}
    Therefore, 
    \begin{eqnindent}
        \begin{flalign}
            \int_S\bm{A} \cdot d\bm{a} = \int_S\bm{A} \cdot \bm{n}da &&
        \end{flalign}
    \end{eqnindent}
    which shows that the integral of $\bm{A}$ over $S$ is the integral of the normal component of $\bm{A}$ over $S$ or
    \begin{eqnindent}
        \begin{flalign}
            \int_S\bm{A} \cdot d\bm{a} = \int_S\sum_iA_ida_i &&
        \end{flalign}
    \end{eqnindent}
    Consider a path extending from point $B$ to $C$ \texttt{(Figure 1.17-2)}\newline
    The line integral along the path from point $B$ to $C$ of a vector function $\bm{A} = \bm{A}\paren{x_i}$ is the integral of the component of $\bm{A}$ along the path
    \begin{eqnindent}
        \begin{flalign}
            \int_{BC}\bm{A} \cdot d\bm{s} = \int_{BC}\sum_iA_idx_i &&
        \end{flalign}
    \end{eqnindent}
    where $d\bm{s}$ is an element of the length with magnitude $ds$ and direction along the direction of the path traversed. \newline
    Consider a closed volume $V$ enclosed by the surface $S$ \texttt{(Figure 1.17-3)}\newline
    Let the vector $\bm{A}$ and its first derivatives be continuous throughout the volume. \newline
    Gauss's theorem (divergence theorem) -- the surface integral of $\bm{A}$ over the closed surface $S$ is equal to the volume integral of the divergence of $\bm{A}$ ($\bm{\nabla} \cdot \bm{A}$) throughout the volume $V$ enclosed by the surface $S$
    \begin{eqnindent}
        \begin{flalign}
            \int_S\bm{A} \cdot d\bm{a} = \int_V\bm{\nabla} \cdot \bm{A}dv &&
        \end{flalign}
    \end{eqnindent}
    Consider an open surface $S$ and the contour path $C$ that defines the surface \texttt{(Figure 1.17-4)}\newline
    Stoke's theorem -- the line integral of the vector $\bm{A}$ around the contour path $C$ is equal to the surface integral of the curl of $\bm{A}$ over the surface defined by $C$
    \begin{eqnindent}
        \begin{flalign}
            \int_C\bm{A} \cdot d\bm{s} = \int_S\paren{\bm{\nabla} \times \bm{A}} \cdot d\bm{a} &&
        \end{flalign}
    \end{eqnindent}
    where the line integral is around the closed contour path $C$ and $\bm{\nabla} \times \bm{A}$ must exist and be integrable over the entire surface $S$. 

\end{document}
