\documentclass[../main.tex]{subfiles}

\begin{document}

    \subsection{Properties of Rotation Matrices}
    Consider a line segment extending in a certain direction in space with the origin of the coordinate system lying at some point on the line \texttt{(Figure 1.4-1)}\newline
    The direction cosines of the line -- cosines of definite angles the line makes with each coordinate axes
    \begin{hookeditemize}
        \item relates to unity as
        \begin{eqnindent}
            \begin{flalign}
                \cos^2\alpha + \cos^2\beta + \cos^2\gamma = 1 &&
            \end{flalign}
        \end{eqnindent}
    \end{hookeditemize}
    Another similar line segment is added such that it intersects the original line segment at the origin \texttt{(Figure 1.4-2)}\newline
    The cosine of the angle $\theta$ between the two lines relates to the direction cosines of both lines as
    \begin{eqnindent}
        \begin{flalign}
            \label{eq:13}
            \cos\theta = \cos\alpha\cos\alpha' + \cos\beta\cos\beta' + \cos\gamma\cos\gamma' &&
        \end{flalign}
    \end{eqnindent}
    \blankline
    Looking at coordinate transformations on a 3-D coordinate system rotated about some axis through the origin. \newline
    We specify the coordinate transformations through the 9 individual elements $\lambda_{ij}$ of the transformation matrix $\bm{\lambda}$
    \begin{dasheditemize}
        \item 6 quantities are related\newline
        Suppose the $x_1'$-, $x_2'$- \& $x_3'$-axes are lines in the $\paren{x_1,~x_2,~x_3}$ coordinate system such that lines $x_1'$, $x_2'$ \& $x_3'$ are defined with direction cosines $\paren{\lambda_{11},~\lambda_{12},~\lambda_{13}}$, $\paren{\lambda_{21},~\lambda_{22},~\lambda_{23}}$ \& $\paren{\lambda_{31},~\lambda_{32},~\lambda_{33}}$ respectively
\pagebreak
        \begin{multicols}{2}
            \vspace{- \bigskipamount}
            For each line,
            \begin{eqnindent}
                \begin{flalign}
                    \begin{rcases}
                        \lambda_{11}^2 + \lambda_{12}^2 + \lambda_{13}^2 = 1\quad\\
                        \lambda_{21}^2 + \lambda_{22}^2 + \lambda_{23}^2 = 1\quad\\
                        \lambda_{31}^2 + \lambda_{32}^2 + \lambda_{33}^2 = 1\quad
                    \end{rcases} &&
                \end{flalign}
            \end{eqnindent}
            which in summation notation surmises to
            \begin{eqnindent}
                \begin{flalign}
                    \sum_j\lambda_{ij}\lambda_{kj} = 1,\quad i = k &&
                \end{flalign}
            \end{eqnindent}
            \blankline
            As each line is perpendicular to each other,
            \begin{eqnindent}
                \begin{flalign}
                    \begin{rcases}
                        \lambda_{11}\lambda_{21} + \lambda_{12}\lambda_{22} + \lambda_{13}\lambda_{23} = \cos\paren{\frac{\pi}{2}} = 0\quad\\
                        \lambda_{21}\lambda_{31} + \lambda_{22}\lambda_{32} + \lambda_{23}\lambda_{33} = \cos\paren{\frac{\pi}{2}} = 0\quad\\
                        \lambda_{11}\lambda_{31} + \lambda_{12}\lambda_{32} + \lambda_{13}\lambda_{33} = \cos\paren{\frac{\pi}{2}} = 0\quad
                    \end{rcases} &&
                \end{flalign}
            \end{eqnindent}
            which in summation notation surmises to
            \begin{eqnindent}
                \begin{flalign}
                    \sum_j\lambda_{ij}\lambda_{kj} = 0,\quad i \neq k &&
                \end{flalign}
            \end{eqnindent}
        \end{multicols}
        \vspace{- \bigskipamount}
        \item 3 independent quantities
    \end{dasheditemize}
    Of the 6 non-independent quantities, we can combine the summation results into
    \begin{eqnindent}
        \begin{flalign}
            \sum_j\lambda_{ij}\lambda_{kj} = \delta_{ik} &&
        \end{flalign}
    \end{eqnindent}
    where $\delta_{ik}$ is the Kronecker delta symbol
    \begin{eqnindent}
        \begin{flalign}
            \hookrightarrow \delta_{ik} = \begin{cases}
                \quad0,\quad i \neq k\\
                \quad1,\quad i = k
            \end{cases} &&
        \end{flalign}
    \end{eqnindent}
    The above result is the orthogonality condition\footnote{
        If instead the $x_1$-, $x_2$- \& $x_3$-axes were taken as lines in the $\paren{x_1',~x_2',~x_3'}$ coordinate system, it will yield the relation $\sum_i\lambda_{ij}\lambda_{ik} = \delta_{ik}$, which is mathematically similar to the result from the former construction
    } -- true when coordinate axes in each of the systems specified in the rotation are mutually perpendicular (orthogonal)
    \blankline
    The transformation of coordinates and properties of transformation matrices are mathematically similar for differing constructions
    \begin{dasheditemize}
        \item the transformation acts on the point $P$, giving a new state of the point (point $P'$) expressed with respect to a fixed coordinate system \texttt{(Figure 1.4-3)}
        \item the transformation acts on the frame of reference \texttt{(Figure 1.4-4)}
    \end{dasheditemize}
    \begin{hookeditemize}
        \item The coordinates of point $P'$ in the former construction is equivalent to the new coordinates $\paren{x_1',~x_2'}$ of point $P$ in the latter construction
    \end{hookeditemize}

\end{document}
