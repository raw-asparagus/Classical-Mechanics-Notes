\documentclass[../main.tex]{subfiles}

\begin{document}

    \subsection{Unit Vectors}
    Unit vetors - a vector with length equal to the unit of length used along the particular coordinate axes
    \begin{indented}
        e.g. variants of the symbol for unit vectors
        \begin{itemize}
            \renewcommand\labelitemi{--}
            \item $\paren{\bm{i},~\bm{j},~\bm{k}}$
            \item $\paren{\bm{e}_1,~\bm{e}_2,~\bm{e}_3}$
            \item $\paren{\bm{e}_r,~\bm{e}_{\theta},~\bm{e}_{\phi}}$
            \item $\paren{\hat{r},~\hat{\theta},~\hat{\phi}}$
        \end{itemize}
        i.e. vector $\bm{A} = \paren{A_1,~A_2,~A_3}$ to be expressed in terms of unit vectors
        \begin{itemize}
            \renewcommand\labelitemi{--}
            \item $\bm{A} = \bm{e}_1A_1 + \bm{e}_2A_2 + \bm{e}_3A_3 = \sum_i\bm{e}_iA_i$
            \item $\bm{A} = A_1\bm{i} + A_2\bm{j} + A_3\bm{k}$
        \end{itemize}
        i.e. unit vector along the radial direction
        \begin{eqnindent}
            \begin{flalign}
                \bm{e}_R = \frac{\bm{R}}{\abs{\bm{R}}} &&
            \end{flalign}
        \end{eqnindent}
    \end{indented}
    The components of a vector can be obtained by projection onto its axes
    \begin{eqnindent}
        \begin{flalign}
            A_i = \bm{e}_i \cdot \bm{A} &&
        \end{flalign}
    \end{eqnindent}
    From the results of a scalar product, for any two orthogonal vectors, 
    \begin{eqnindent}
        \begin{flalign}
            \bm{e}_i \cdot \bm{e}_j = \abs{\bm{e}_i}\abs{\bm{e}_j}\cos\paren{i,~j} = \delta_{ij} &&
        \end{flalign}
    \end{eqnindent}

\end{document}
