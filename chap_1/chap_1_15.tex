\documentclass[../main.tex]{subfiles}

\begin{document}

    \subsection{Angular Velocity}
    Consider a particle moving in \texttt{(Figure 1.15-1)} in a cylindrical coordinate system with the origin lying at an arbitrary point $O$ on the axis of rotation\newline
    The path the particle describes in an infinitesimal time $\delta t$ may be represented as an infinitesimal arc of a circle with the instantaneous axis of rotation being the line passing through the center of the circle perpendicular to the instantaneous direciton of motion\newline
    As the position vector of a point changes from $\bm{r}$ to $\bm{r} + \delta\bm{r}$, 
    \begin{eqnindent}
        \begin{flalign}
            \label{eq:95}
            \delta\bm{r} = \delta\bm{\theta} \times \bm{r} &&
        \end{flalign}
    \end{eqnindent}
    where the quantity $\delta\bm{\theta}$ has magnitude equal to the infinitesimal rotation angle and direction along the instantaneous axis of rotation\newline
    Only such infinitesimal rotations can be represented by vectors
    \begin{dasheditemize}
        \item Finite rotations -- do not commute; gives different results depending on order of rotations\newline
        i.e.
        \begin{indented}
            \begin{eqnindent}
                \begin{flalign}
                    \begin{split}
                        \bm{\lambda}_3 = \bm{\lambda}_1\bm{\lambda}_2\quad&\implies\quad\bm{C} = \bm{A} + \bm{B}\\
                        \bm{\lambda}_4 = \bm{\lambda}_2\bm{\lambda}_1\quad&\implies\quad\bm{D} = \bm{B} + \bm{A}
                    \end{split} &&
                \end{flalign}
            \end{eqnindent}
            In general $\bm{\lambda}_3 \not\equiv\ \bm{\lambda}_4$ even though $\bm{C} \equiv \bm{D}$
            \begin{hookeditemize}
                \item successive applications of finite rotations do not commute
            \end{hookeditemize}
        \end{indented}
        \item Infinitesimal rotations -- do commute\newline
        i.e.
        \begin{indented}
            Suppose a point moves as in rotations $\delta\bm{\theta}_1$ and $\delta\bm{\theta}_2$ \texttt{(Figure 1.15-2)}
            \vspace{- \bigskipamount}
            \begin{multicols}{2}
                As the rotation $\delta\bm{\theta}_1$ takes $\bm{r}$ into $\bm{r} + \delta\bm{r}_1$, 
                \begin{eqnindent}
                    \begin{flalign}
                        \delta\bm{r}_1 = \delta\bm{\theta}_1 \times \bm{r} &&
                    \end{flalign}
                \end{eqnindent}
                a successive rotaton $\delta\bm{\theta}_2$ from $\bm{r} + \delta\bm{r}_1$ will result in
                \begin{eqnindent}
                    \begin{flalign}
                        \delta\bm{r}_2 = \delta\bm{\theta}_2 \times \paren{\bm{r} + \delta\bm{r}_1} &&
                    \end{flalign}
                \end{eqnindent}
                which gives 
                \begin{eqnindent}
                    \begin{flalign}
                        \begin{split}
                            \bm{r} + \delta\bm{r}_{12} &= \bm{r} + \sparen{\delta\bm{\theta}_1 \times \bm{r} + \delta\bm{\theta}_2 \times \paren{\bm{r} + \delta\bm{r}_1}}\\
                            \delta\bm{r}_{12} &\approx \delta\bm{\theta}_1 \times \bm{r} + \delta\bm{\theta}_2 \times \bm{r}
                        \end{split} &&
                    \end{flalign}
                \end{eqnindent}
                Or instead if the rotation $\delta\bm{\theta}_2$ suceeds $\delta\bm{\theta}_1$, 
                \begin{eqnindent}
                    \begin{flalign}
                        \begin{split}
                            \delta\bm{r}_2 &= \delta\bm{\theta}_2 \times \bm{r}\\
                            \delta\bm{r}_1 &= \delta\bm{\theta}_1 \times \paren{\bm{r} + \delta\bm{r}_2}
                        \end{split} &&
                    \end{flalign}
                \end{eqnindent}
                which gives
                \begin{eqnindent}
                    \begin{flalign}
                        \begin{split}
                            \bm{r} + \delta\bm{r}_{21} &= \bm{r} + \sparen{\delta\bm{\theta}_2 \times \bm{r} + \delta\bm{\theta}_1 \times \paren{\bm{r} + \delta\bm{r}_2}}\\
                            \delta\bm{r}_{21} &\approx \delta\bm{\theta}_2 \times \bm{r} + \delta\bm{\theta}_1 \times \bm{r}
                        \end{split} &&
                    \end{flalign}
                \end{eqnindent}
            \end{multicols}
            \vspace{- \bigskipamount}
            Comparing both sets of successive rotations, $\delta\bm{r}_{12} \equiv \delta\bm{r}_{21}$
            \begin{hookeditemize}
                \item successive applications of infinitesimal rotations do commute
            \end{hookeditemize}
        \end{indented}
    \end{dasheditemize}
    The time rate of change of the position vector is the linear velocity vector of the particle
    \begin{indented}
        i.e.
        \begin{eqnindent}
            \begin{flalign}
                \dot{\bm{r}} = \bm{v} &&
            \end{flalign}
        \end{eqnindent}
    \end{indented}
    \begin{dasheditemize}
        \item magnitude
        \begin{eqnindent}
            \begin{flalign}
                \begin{split}
                    v = R\frac{d\theta}{dt} &= R\omega,\quad\text{where }R = r\sin\alpha\\
                    &= r\omega\sin\alpha
                \end{split} &&
            \end{flalign}
        \end{eqnindent}
        \item direction perpendicular to $\bm{r}$ on the plane of the circle
    \end{dasheditemize}
    Angular velocity vector $\bm{\omega}$ -- rate of change of the angular position vector
    \begin{dasheditemize}
        \item defined along the normal of the plane with the positive direction corresponding to the direction of advance of a right-hand screw when turned in the same sense as the rotation of the particle
        \item described as the ratio of an infinitesimal rotation angle to an infinitesimal time\newline
        i.e.
        \begin{eqnindent}
            \begin{flalign}
                \bm{\omega} = \frac{\delta\bm{\theta}}{\delta t} &&
            \end{flalign}
        \end{eqnindent}
    \end{dasheditemize}
    From equation (\ref{eq:95}),
    \begin{eqnindent}
        \begin{flalign}
            \begin{split}
                \frac{\delta\bm{r}}{\delta t} &= \frac{\delta\bm{\theta}}{\delta t} \times \bm{r}\\
                \implies \bm{v} &= \bm{\omega} \times \bm{r},\quad\text{as }\delta t \rightarrow 0
            \end{split} &&
        \end{flalign}
    \end{eqnindent}

\end{document}
